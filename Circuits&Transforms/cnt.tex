

\documentclass[journal,12pt,twocolumn]{IEEEtran}
%
\usepackage{setspace}
\usepackage{gensymb}
\usepackage{xcolor}
\usepackage{caption}
%\usepackage{subcaption}
%\doublespacing
\singlespacing

%\usepackage{graphicx}
%\usepackage{amssymb}
%\usepackage{relsize}
\usepackage[cmex10]{amsmath}
\usepackage{mathtools}
\usepackage{circuitikz}
%\usepackage{amsthm}
%\interdisplaylinepenalty=2500
%\savesymbol{iint}
%\usepackage{txfonts}
%\restoresymbol{TXF}{iint}
%\usepackage{wasysym}
\usepackage{hyperref}
\usepackage{amsthm}
\usepackage{mathrsfs}
\usepackage{txfonts}
\usepackage{stfloats}
\usepackage{cite}
\usepackage{cases}
\usepackage{subfig}
%\usepackage{xtab}
\usepackage{longtable}
\usepackage{multirow}
%\usepackage{algorithm}
%\usepackage{algpseudocode}
%\usepackage{enumerate}
\usepackage{enumitem}
\usepackage{mathtools}
%\usepackage{iithtlc}
%\usepackage[framemethod=tikz]{mdframed}
\usepackage{listings}
\let\vec\mathbf


%\usepackage{stmaryrd}


%\usepackage{wasysym}
%\newcounter{MYtempeqncnt}
\DeclareMathOperator*{\Res}{Res}
%\renewcommand{\baselinestretch}{2}
\renewcommand\thesection{\arabic{section}}
\renewcommand\thesubsection{\thesection.\arabic{subsection}}
\renewcommand\thesubsubsection{\thesubsection.\arabic{subsubsection}}

\renewcommand\thesectiondis{\arabic{section}}
\renewcommand\thesubsectiondis{\thesectiondis.\arabic{subsection}}
\renewcommand\thesubsubsectiondis{\thesubsectiondis.\arabic{subsubsection}}

%\renewcommand{\labelenumi}{\textbf{\theenumi}}
%\renewcommand{\theenumi}{P.\arabic{enumi}}

% correct bad hyphenation here
\hyphenation{op-tical net-works semi-conduc-tor}

\lstset{
language=Python,
frame=single, 
breaklines=true,
columns=fullflexible
}



\begin{document}
%

\theoremstyle{definition}
\newtheorem{theorem}{Theorem}[section]
\newtheorem{problem}{Problem}
\newtheorem{proposition}{Proposition}[section]
\newtheorem{lemma}{Lemma}[section]
\newtheorem{corollary}[theorem]{Corollary}
\newtheorem{example}{Example}[section]
\newtheorem{definition}{Definition}[section]
%\newtheorem{algorithm}{Algorithm}[section]
%\newtheorem{cor}{Corollary}
\newcommand{\BEQA}{\begin{eqnarray}}
\newcommand{\EEQA}{\end{eqnarray}}
\newcommand{\define}{\stackrel{\triangle}{=}}
\newcommand{\myvec}[1]{\ensuremath{\begin{pmatrix}#1\end{pmatrix}}}
\newcommand{\mydet}[1]{\ensuremath{\begin{vmatrix}#1\end{vmatrix}}}
\bibliographystyle{IEEEtran}
%\bibliographystyle{ieeetr}
\providecommand{\nCr}[2]{\,^{#1}C_{#2}} % nCr
\providecommand{\nPr}[2]{\,^{#1}P_{#2}} % nPr
\providecommand{\mbf}{\mathbf}
\providecommand{\pr}[1]{\ensuremath{\Pr\left(#1\right)}}
\providecommand{\qfunc}[1]{\ensuremath{Q\left(#1\right)}}
\providecommand{\sbrak}[1]{\ensuremath{{}\left[#1\right]}}
\providecommand{\lsbrak}[1]{\ensuremath{{}\left[#1\right.}}
\providecommand{\rsbrak}[1]{\ensuremath{{}\left.#1\right]}}
\providecommand{\brak}[1]{\ensuremath{\left(#1\right)}}
\providecommand{\lbrak}[1]{\ensuremath{\left(#1\right.}}
\providecommand{\rbrak}[1]{\ensuremath{\left.#1\right)}}
\providecommand{\cbrak}[1]{\ensuremath{\left\{#1\right\}}}
\providecommand{\lcbrak}[1]{\ensuremath{\left\{#1\right.}}
\providecommand{\rcbrak}[1]{\ensuremath{\left.#1\right\}}}
\theoremstyle{remark}
\newtheorem{rem}{Remark}
\newcommand{\sgn}{\mathop{\mathrm{sgn}}}
\providecommand{\abs}[1]{\left\vert#1\right\vert}
\providecommand{\res}[1]{\Res\displaylimits_{#1}} 
\providecommand{\norm}[1]{\lVert#1\rVert}
\providecommand{\mtx}[1]{\mathbf{#1}}
\providecommand{\mean}[1]{E\left[ #1 \right]}
\providecommand{\fourier}{\overset{\mathcal{F}}{ \rightleftharpoons}}
\providecommand{\ztrans}{\overset{\mathcal{Z}}{ \rightleftharpoons}}
%\providecommand{\hilbert}{\overset{\mathcal{H}}{ \rightleftharpoons}}
\providecommand{\system}{\overset{\mathcal{H}}{ \longleftrightarrow}}
	%\newcommand{\solution}[2]{\textbf{Solution:}{#1}}
\newcommand{\solution}{\noindent \textbf{Solution: }}
\providecommand{\dec}[2]{\ensuremath{\overset{#1}{\underset{#2}{\gtrless}}}}
\numberwithin{equation}{section}
%\numberwithin{equation}{subsection}
%\numberwithin{problem}{subsection}
%\numberwithin{definition}{subsection}
\makeatletter
\@addtoreset{figure}{problem}
\makeatother
\let\StandardTheFigure\thefigure
%\renewcommand{\thefigure}{\theproblem.\arabic{figure}}
\renewcommand{\thefigure}{\theproblem}
%\numberwithin{figure}{subsection}
\def\putbox#1#2#3{\makebox[0in][l]{\makebox[#1][l]{}\raisebox{\baselineskip}[0in][0in]{\raisebox{#2}[0in][0in]{#3}}}}
     \def\rightbox#1{\makebox[0in][r]{#1}}
     \def\centbox#1{\makebox[0in]{#1}}
     \def\topbox#1{\raisebox{-\baselineskip}[0in][0in]{#1}}
     \def\midbox#1{\raisebox{-0.5\baselineskip}[0in][0in]{#1}}
\vspace{3cm}
\title{ 
%\logo{
%}
Circuits and Transforms
%	\logo{Octave for Math Computing }
}

\author{ AI21BTECH11016 - Blessy Anvitha J}

\maketitle
%\newpage
\tableofcontents
%\renewcommand{\thefigure}{\thesection.\theenumi}
%\renewcommand{\thetable}{\thesection.\theenumi}
\renewcommand{\thefigure}{\theenumi}
\renewcommand{\thetable}{\theenumi}
%\renewcommand{\theequation}{\thesection}
\bigskip


%%\includegraphics[scale=0.95]{Yellow-Line}
%\begin{tikzpicture}
%\definecolor{yellow1}{rgb}{0.95,0.77,0.2}
%\draw[line width=0.75mm, yellow1] (0,0) -- (\textwidth,0);
%\end{tikzpicture}
%\par\end{center}
\section{Definitions}
\begin{enumerate}[label=\arabic*.,ref=\thesection.\theenumi]
\numberwithin{equation}{section}
\numberwithin{figure}{section}
\item The unit step function is 
\begin{align}
\label{unit - step}
u(t) =
\begin{cases}
1 & t > 0
\\
	\frac{1}{2} & t = 0
\\
0 & t < 0
\end{cases}
\end{align}
\item The Laplace transform of $g(t)$ is defined as 
\begin{align}
	G(s) = \int_{-\infty}^{\infty} g(t) e^{-st}\, dt
\end{align}
\end{enumerate}
\section{Laplace Transform}
\begin{enumerate}[label=\arabic*.,ref=\thesection.\theenumi]
\numberwithin{equation}{section}
\item In the circuit, the switch S is connected to position P for a long time so that the charge on the capacitor
	becomes $q_1 \, \mu C$. Then S is switched to position Q.  After a long time, the charge on the capacitor is $q_2 \, \mu C$.
\item Draw the circuit using latex-tikz.\\
\solution The following code plots Fig .\ref{fig:ckt}
\begin{lstlisting}
https://github.com/JBA-12/EE3900/blob/main/Circuits%26Transforms/TikzCircuits/2.2.tex
\end{lstlisting}
\begin{figure}[!ht]
\centering
% \begin{circuitikz}[scale=0.7]
% 
%	\draw (0,0) to [battery1,l_=$1V$] ++(0,-3)
%	
%	to ++(11.5,0) to [battery1,l_=$2V$,invert]++(0,3)
%	
%	to [american resistor,l_=$2\Omega$] ++(-6,0)
%	
%	to [capacitor,l_=$1 \mu F$] ++(0,-3);
%	
%	\draw (0,0) to ++(0.7,0) node[anchor=north east]{P};
%	
%	\draw (1,-3) to ++(0,2.6)node[anchor=north west]{Q};
%	
%	\draw (4.5,0) to ++(1.5,0);
%	
%	\draw (4.5,0) to [american resistor,l_=$1\Omega$]++(-2,0) 
%	
%	to ++(-0.8,0.5)node[anchor=south west]{S};
%	
%\end{circuitikz}
\begin{circuitikz}[scale=0.7] 
\draw (0,0) to [battery1,l_=$1V$] ++(0,-3)
	
	to ++(11.5,0) to [battery1,l_=$2V$,invert]++(0,3)
	
	to [american resistor,l_=$2\Omega$] ++(-6,0)
	
	to [capacitor,l_=$1 \mu F$] ++(0,-3);
	
	\draw (0,0) to ++(0.7,0) node[anchor=north east]{P};
	
	\draw (1,-3) to ++(0,2.6)node[anchor=north west]{Q};
	
	\draw (4.5,0) to ++(1.5,0);
	
	\draw (4.5,0) to [american resistor,l_=$1\Omega$]++(-3,0) 
	
	to ++(-0.8,0.4)node[anchor=south west]{S};
;
\end{circuitikz}

\caption{Given Circuit}
\label{fig:ckt}
\end{figure}
\item Find $q_1$.\\
\solution The equivalent circuit in the steady state is as shown in Fig .\ref{fig:ckt-q1}.
\begin{figure}[!ht]
\begin{circuitikz}[scale=0.7]
	\draw (0,0) to [battery1,l_=$1V$] ++(0,-3)
	to ++(11.5,0) to [battery1,l_=$2V$,invert]++(0,3)
	to [american resistor,l_=$2\Omega$] ++(-6,0)
	to [capacitor,l_=$1 \mu F$] ++(0,-3);
	\draw (0,0) to ++(2.5,0);
	\draw (4.5,0) to ++(1.5,0);
	\draw (4.5,0) to [american resistor,l_=$1\Omega$]++(-2,0) ;
\end{circuitikz}
   
\caption{Before switching S to Q}
\label{fig:ckt-q1}
\end{figure}
By applying KVL,
\begin{align}
1-i-2i-2=0\\
3i=-1 \Rightarrow i=\frac{-1}{3}A
\end{align}
Potential Difference across the capacitor at steady state is
\begin{align}
1-\brak{\frac{-1}{3}}=\frac{4}{3}V\\
q_1=\frac{4}{3} \cdot 1\\
=\frac{4}{3} \mu C
\end{align}
\item Show that the Laplace transform of $u(t)$ is $\frac{1}{s}$ and find the ROC.\\
\solution We know that Laplace Transform fo function $f(t)$ is given as $F(s)$,
\begin{align}
\label{eq:LaplaceTrans}
F(s)&= \int_{0}^{\infty} f(t)e^{-st} \,dt \\
\end{align}
For $u(t)$, we have,
\begin{align}
F(s)&=\int_{0}^{\infty} u(t)e^{-st} \,dt
\end{align}
Using \eqref{unit - step}
\begin{align}
F(s)&=\int_{0}^{\infty} u(t)e^{-st} \,dt\\
&=\int_{0}^{\infty} e^{-st} \,dt\\
&=-\brak{0-\frac{1}{s}}\\
&=\frac{1}{s}
\end{align}
ROC is $ Re(s)>0$ since for $s>0$, $e^{-st}<\infty$ for $t \to \infty$
\begin{figure}[!ht]
\centering
\includegraphics[width=\columnwidth]{figs/2.4}
\caption{}
\label{fig:roc1}
\end{figure}
\item Show that 
\begin{align}
e^{-at}u(t) \system{L} \frac{1}{s+a}, \quad a > 0
\end{align}
and find the ROC.\\
\solution From \ref{eq:LaplaceTrans},
\begin{align}
F(s)&=\int_{0}^{\infty} u(t)e^{-at}e^{-st} \,dt\\
&=\int_{0}^{\infty} u(t)e^{-\brak{s+a}t} \,dt\\
&=\int_{0}^{\infty} e^{-\brak{s+a}t} \,dt\\
&=-\brak{0-\frac{1}{s+a}}\\
&=\frac{1}{s+a}
\end{align}
ROC is
\begin{align}
Re(s)+a>0 \Rightarrow  Re(s)>-a
\end{align}
\begin{figure}[!ht]
\centering
\includegraphics[width=\columnwidth]{figs/2.5.png}
\caption{}
\label{fig:roc2}
\end{figure}
\item Now consider the following resistive circuit transformed from 
Fig. \ref{fig:ckt}
\begin{figure}[!ht]
\centering
\includegraphics[width=\columnwidth]{figs/lap-ckt.jpg}
\caption{}
\label{fig:lap-ckt}
\end{figure}
where 
\begin{align}
u(t) \system{L} V_1(s)\\
2u(t) \system{L} V_2(s)
\end{align}
Find the voltage across the capacitor $V_{C_0}(s)$.\\
\solution
\begin{align}
R_{eff}=\frac{1}{1+\frac{1}{2}} = \frac{2}{3} \Omega\\
V_{eff}=\frac{1}{1+\frac{1}{2}} = \frac{2}{3}V
\end{align}
%Effective Circuit in Laplacian Space is
\begin{align}
V_{C_0}(s)&=V_{S}(s)\frac{C_{0}}{C_{0}+R_{eff}}\\
&=\brak{\frac{4}{3s}}\brak{\frac{\frac{1}{s}}{\frac{1}{s}+\frac{2}{3}}}\\
\label{eq:laptr}
&=\frac{3+4s}{3s\brak{s+\frac{3}{2}}}
\end{align}
\item Find $v_{C_0}(t)$.  Plot using python.\\
\solution The following python code gives the plot
\begin{lstlisting}
https://github.com/JBA-12/EE3900/blob/main/Circuits%26Transforms/codes/2.7.py
\end{lstlisting}
and run the code using the following command 
\begin{lstlisting}
python3 2.7.py
\end{lstlisting}
Using \eqref{eq:laptr},
\begin{align}
\frac{3+4s}{3s\brak{s+\frac{3}{2}}}&=\frac{2}{3s}+\frac{2}{3(\frac{3}{2}+s)}
\end{align}
Apply inverse Laplacian Transform,
\begin{align}
V_{C_0}(s)\system{L^{-1}}V_{C_0}(t)\\
\laplaceinv{V_{C_0}(s)}&=\laplaceinv{\frac{2}{3s}+\frac{2}{3(\frac{3}{2}+s)}}\\
&=	\laplaceinv{\frac{2}{3s}}-\frac{2}{3}\laplaceinv{\frac{1}{\frac{3}{2}+s}}
\end{align}
Since,
\begin{align}
\laplaceinv{\frac1s}&=u(t)\\
\laplaceinv{\frac{1}{s-a}}&=e^{at}u(t)
\end{align}
Using the above equations,
\begin{align}
V_{C_0}(t)=\frac{2}{3}\brak{ 1+e^{\frac{-3}{2} t}}u(t)
\end{align}
\begin{figure}[!ht]
\centering
\includegraphics[width=\columnwidth]{figs/2.7.png}
\caption{Plot of $V_{C_0}(t)$}
\label{fig:lap}
\end{figure}
\item Verify your result using ngspice.\\
\solution The following codes verifies results using ngspice
\begin{lstlisting}
https://github.com/JBA-12/EE3900/blob/main/Circuits%26Transforms/codes/2.8.cir
\end{lstlisting}
\begin{lstlisting}
https://github.com/JBA-12/EE3900/blob/main/Circuits%26Transforms/codes/2.8.py
\end{lstlisting}
\begin{figure}[!ht]
\centering
\includegraphics[width=\columnwidth]{figs/2.8.png}
\caption{}
\label{fig:ngspice}
\end{figure}
\item Obtain Fig. 
\ref{fig:lap}
using the equivalent differential equation.\\
\solution Using Kirchoff's junction law
\begin{align}
\frac{v_c(t) - v_1(t)}{R_1} + \frac{v_c(t) - v_2(t)}{R_2} + \frac{\der{q}}{\der{t}} = 0
\end{align}
where $q(t)$ is the charge on the capacitor.\\
On taking the Laplace transform on both sides of this equation
\begin{align}
\frac{V_c(s) - V_1(s)}{R_1} + \frac{V_c(s) - V_2(s)}{R_2} + \brak{sQ(s) - q(0^-)} = 0
\end{align}
But $q(0^-) = 0$ and 
\begin{align}
q(t) &= C_0v_c(t) \\
\implies Q(s) &= C_0V_c(s)
\end{align}
Thus
\begin{align}
&\frac{V_c(s) - V_1(s)}{R_1} + \frac{V_c(s) - V_2(s)}{R_2} + sC_0V_c(s) = 0 \\
\implies &\frac{V_c(s) - V_1(s)}{R_1} + 	\frac{V_c(s) - V_2(s)}{R_2} + \frac{V_c(s) - 0}{\frac{1}{sC_0}} = 0 
\end{align}
which is the same equation as the one we obtained from Fig. \ref{fig:lap}
\end{enumerate}
\section{Initial Conditions}
\begin{enumerate}[label=\arabic*.,ref=\thesection.\theenumi]
\numberwithin{equation}{section}
\item Find $q_2$ in Fig. \ref{fig:ckt}.\\
\solution At steady state capacitor behaves as an open switch. Hence $V_{C_0}=V_{1 \Omega}$.\\
Let $i$ be the current in the circuit. Using KVL,
\begin{align}
2-2i-i=0 \implies i=\frac{2}{3}\\
V_{1 \Omega}=i \times 1= \frac{2}{3} V\\
V_{C_0}=\frac{q_2}{C_0}=V_{1 \Omega}=\frac{2}{3}\\
\implies q_2=\frac{2}{3} \mu C
\end{align}
\item Draw the equivalent $s$-domain resistive circuit when S is switched to position Q.  Use variables $R_1, R_2, C_0$ for the passive elements.
Use latex-tikz.
\label{prob:init}\\
\solution 
\begin{figure}[!ht]
\centering
    \begin{circuitikz} 
    \ctikzset{resistor = european}
    \draw
    (0,0) -- (0,3)
    node[label={above:Q}] {}
    to[R, l^=$R_1$, *-*] (3,3) 
    node[label={above:X}] {}
    to[R, l^=$R_2$] (5.5,3)
    to[battery1, l= $\frac{2}{s} V$] (5.5,0)
    -- (0,0)
    (3,3) to[battery1, l=$\frac{4}{3s} V$] (3,2) to[R, l=$\frac{1}{sC_0}$] (3,0) 
    -- (3,-0.5) node[ground, label={right:G}] {};
    \end{circuitikz}

\caption{After switching S to Q}
\label{fig:sq}
\end{figure}
\item $V_{C_0}(s)$ = ?  \\
\solution Let voltage across capacitor be $V$. Using KCL at node in Fig. \ref{fig:sq}
\begin{align}
\frac{V - 0}{R_1} + \frac{V - \frac{2}{s}}{R_2} + sC_0\brak{V - \frac{4}{3s}} = 0 \\
\implies V_{C_0}(s) = \frac{\frac{2}{sR_2} + \frac{4C_0}{3}}{\frac{1}{R_1} + \frac{2}{R_2} + sC_0}
\label{eq:v2-s}
\end{align} 
\item $v_{C_0}(t)$ = ? Plot using python.\\
\solution From \eqref{eq:v2-s},
\begin{align}
&V_{C_0}(s) = \frac{4}{3}\brak{\frac{1}{\frac{1}{C_0}\brak{\frac{1}{R_1} + \frac{1}{R_2}}+s}} \nonumber \\
&+ \frac{2}{R_2\brak{\frac{1}{R_1} +\frac{1}{R_2}}}\brak{\frac{1}{s} - \frac{1}{\frac{1}{C_0}\brak{\frac{1}{R_1} + \frac{1}{R_2}} + s}}
\end{align}
Taking an inverse Laplace Transform,
\begin{align}
&v_{C_0}(t) = \frac{4}{3}e^{-\brak{\frac{1}{R_1} + \frac{1}{R_2}}\frac{t}{C_0}}u(t) \nonumber \\ 
&+ \frac{2}{R_2\brak{\frac{1}{R_1}+\frac{1}{R_2}}}\brak{1 - e^{-\brak{\frac{1}{R_1} + \frac{1}{R_2}}\frac{t}{C_0}}}u(t)
\end{align}
Substituting values gives
\begin{align}
v_{C_0}(t) = \frac{2}{3}\brak{1 +e^{-\brak{1.5 \times 10^6}t}}u(t)
\label{eq:v2-t}
\end{align}
\begin{figure}[!ht]
\centering
\includegraphics[width=\columnwidth]{figs/3.4.png}
\caption{Plot of $V_{C_0}(t)$}
%\label{fig:lap-ckt}
\end{figure}
The following python code gives the plot
\begin{lstlisting}
https://github.com/JBA-12/EE3900/blob/main/Circuits%26Transforms/codes/3.4.py
\end{lstlisting}
\item Verify your result using ngspice.\\
\solution The following codes verifies and plots the results
\begin{lstlisting}
https://github.com/JBA-12/EE3900/blob/main/Circuits%26Transforms/codes/3.5.cir
\end{lstlisting}
\begin{lstlisting}
https://github.com/JBA-12/EE3900/blob/main/Circuits%26Transforms/codes/3.5.py
\end{lstlisting}
\begin{figure}[!ht]
\centering
\includegraphics[width=\columnwidth]{figs/3.5.png}
\caption{ngspice plot of $V_{C_0}(t)$} 
\label{fig:ngspice2}
\end{figure}
\item Find $v_{C_0}(0-), v_{C_0}(0+)$ and  $v_{C_0}(\infty) $. \\
\solution From the initial conditions,
\begin{align}
v_{C_0}(0-) = \frac{q_1}{C} = {\frac{4}{3}}{V}
\end{align}
Using \eqref{eq:v2-t},
\begin{align}
v_{C_0}(0+) &= \lim_{t \to 0+}v_{C_0}(t) = {\frac{4}{3}}{V} \\
v_{C_0}(\infty) &= \lim_{t \to \infty}v_{C_0}(t) = {\frac{2}{3}}{V}
\end{align}
\item Obtain the Fig.  in problem 
\ref{prob:init}
using the equivalent differential equation.\\
\solution Using Kirchoff's junction law
\begin{align}
\label{eq:a}
\frac{v_c(t) - 0}{R_1} + \frac{v_c(t) - v_2(t)}{R_2} + \frac{\der{q}}{\der{t}} = 0
\end{align}
where $q(t)$ is the charge on the capacitor.
On taking the Laplace transform on both sides of the equation \eqref{eq:a}, we get,
\begin{align}
\frac{V_c(s) - 0}{R_1} + \frac{V_c(s) - V_2(s)}{R_2} +sQ(s) - q(0^-) = 0
\end{align}
But $q(0^-) = \frac43 C_0$ and 
\begin{align}
q(t) &= C_0v_c(t) \\
\implies Q(s) &= C_0V_c(s)
\end{align}
Thus
\begin{align}
&\frac{V_c(s) - 0}{R_1} + \frac{V_c(s) - V_2(s)}{R_2} + \brak{sC_0V_c(s) - \frac43 C_0} = 0 \\
\implies &\frac{V_c(s) - 0}{R_1} + 	\frac{V_c(s) - V_2(s)}{R_2} + \frac{V_c(s) - \frac{4}{3s}}{\frac{1}{sC_0}} = 0 
\end{align}
which is the same equation as the one we obtained from Fig. \ref{prob:init}
\end{enumerate}
\section{Bilinear Transform}
\begin{enumerate}[label=\arabic*.,ref=\thesection.\theenumi]
\numberwithin{equation}{section}
\item In Fig. 
\ref{fig:ckt},
consider the case when $S$ is switched to $Q$ right in the beginning. Formulate the differential equation.
\item Find $H(s)$ considering the ouput voltage at the capacitor.
\item Plot $H(s)$.  What kind of filter is it?
\item Using trapezoidal rule for integration, formulate the difference equation by considering 
\begin{align}
y(n) = y(t)\vert_{t=n}
\end{align}
\item Find $H(z)$.
\item How can you obtain $H(z)$ from $H(s)$?
\end{enumerate}
\end{document}